\chapter{Synthesis}
The design was synthesized using the Xilinx ISE suite and the \em hdllab\em \ configuration. 
The target device for synthesis is a Xilinx Virtex-5 FPGA. Since MIPS is a rather compact architecture, the target FPGA provides 
a lot of programmable resources and a clock frequency of 50 MHz is a mandatory requirement, we opted for the design strategy that 
promised the highest maximum clock frequency and did not care much about the used space. However a comparison to the results of the 
balanced synthesis strategy showed no significant differences.

The synthesis result can be operated with a clock frequency up to 73.27 MHz and thus with a minimal clock period of 13.65 ns, 
so the frequency requirements to the project can be fulfilled. The maximum delay is caused by the instruction decode stage, 
which is the most complex part of the processor and it is not surprising that it is limiting the maximum frequency of the whole design. 
The minimum input arrival time of the design is 9.82 ns, the maximum output time required after clock cycle is 3.27 ns and the maximum combinatorial path delay is 1.15 ns.

Although the resource usage was not an optimization goal, the synthesis result is rather compact. It uses 3579 slice registers and 4090 slice look-up-tables, 
which is 7\% of the FPGA's total available slice registers and 9\% of the look-up-tables. Combined, 7281 logic slices are used by the design. 
388 of these slices are fully used, 3702 of them contain an unused flip-flop and 3191 contain an unused look-up-table. This means that 95\% of the used slices are not fully utilized. 
An area-optimizing synthesis strategy could improve this rather bad utilization percentage values but the design still takes only 16 \% of the available logic slices on the FPGA so 
there is no urgent reason to shrink the size of the result at the cost of making it slower. In fact, our design could surely be used on a more low-end FPGA at about the same speed. 
On the other hand the free space on the Virtex-5 FPGA could easily be used to build a MIPS-based System-on-chip on top of our project. Of course for full MIPS compatibility the control 
path and the ALU have to be expanded, but there is plenty of space for it.

The cpu uses 13 IO ports of the FPGA. That makes eight for the LEDs, two for UART, two more for the clock and one for the reset. Of the available 640 IOBs, the 13 used ones make only 2\%.

The synthesis result was tested with the original counter assembler code, waiting 16 clock cycles before the memory cell is incremented by one. The UART-interface was compiled with the frequency parameter configured for 50 MHz to meet the actual board's clock frequency and the \em hdllab\em -project uses the given memory for FPGAs, not the memory used for simulation. The LEDs flashed as they were supposed to be but 16 clock cycles are to fast for a human eye to detect, so we used an attached oscilloscope to test for the correct blinking sequence. The counter version that should make the LEDs blink perceivably for the human eye however did not work. We suppose that this modified counter contains MIPS-instructions that are somehow not supported by our implementation. Although we searched for new instructions and added code to the controller (the datapath should be able to process any MIPS-instruction and needs no further modification) to process them, the example did not make the LEDs blink. But since the requirement to run the original counter was fulfilled and the time was pressing, no further efforts were made to make this functionality possible.