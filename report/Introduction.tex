\chapter{Introduction}
In the HDL Lab is a practical exercise of a hdl implementation. This semester the task is the implementation of a MIPS I microcontroller in vhdl. 
A requirement to this laboratory is the lecture HDL: Verilog and VHDL by Prof. Dr.-Ing. Klaus Hofmann.\\
MIPS is an acronym for Microprocessor without interlocked pipline stages.
The MIPS instruction set is a reduced instruction set computer (RISC). There are available references for 32 and 64-bit with many revisions.\\
This structure was developed in the 80s with the intent to take fully advantage of pipelines. Nowadays this instruction set and structure 
is often used as an hdl first project. Commercially it is used embedded systems such as Windows CE devices, routers, residential gateways and video
consoles such as Nintendo 64, Sony Playstation, Playstation 2 and Playstation Portable. \\

\section{Task}

The objective is to design, implement, synthesise and test a MIPS-I specified processor core on FPGA. 
The hardware description langugage is VHDL and the target technology is a Virtex5 from Xilinx.
The synthesised microcontroller must be able to run at 50 MHz with a desirable frequency of 200 MHz. 
The microcontroller must use a pipeline of a minimum of 2 and maximum of 6 stages.
The following subcomponents are mandatory: ALU, datapath and controlpath.\\
A counter test program shall run on the synthesised microcontroller, outputing the counter value
to eight LEDs on the FPGA board.
