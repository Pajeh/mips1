\documentclass[oneside,colorback,accentcolor=tud4c,11pt,big chapter]{tudreport}
 \renewcommand{\baselinestretch}{1.0}\normalsize


%---- Laden der Pakete  ----%
\usepackage[stable]{footmisc}
\usepackage{booktabs}
\usepackage{multirow}
\usepackage{longtable}

%Standardpakete
\usepackage[utf8,utf8x]{inputenc}
\usepackage[T1]{fontenc}
\usepackage[english]{babel} 
\usepackage{subcaption}

%Matheumgebung
\usepackage{amsmath}

%Bilder und Bildunterschriften
\usepackage{graphicx}
\usepackage[section]{placeins}
\usepackage{caption}
\usepackage{float}
\usepackage{subcaption}
\usepackage{geometry}
\usepackage[utf8]{inputenc}
\usepackage{transparent}
\usepackage{epstopdf}
 
%Tabellen
\usepackage{tabularx}
\usepackage{booktabs}
\usepackage{threeparttable}
\usepackage{multirow}

%Querverweise und Literaturverzeichnis
\usepackage{hyperref}
%\usepackage[pagebackref]{hyperref}
\usepackage[english,compress]{cleveref}
\usepackage{breakurl,url}
\usepackage[numbers,comma,compress]{natbib} %bei bibliographystyle{natdin} noetig

\usepackage{braket}
%Farben
\usepackage{color}
\usepackage{transparent}
\usepackage{epstopdf}

%physikalische und chemische Extras und Einheiten
\usepackage{upgreek}
\usepackage{siunitx}
\sisetup{redefine-symbols=false}
\usepackage{nicefrac}

  \usepackage{multirow} 
  \usepackage{pgfplots}
  \usepackage{multicol} 
  \usepackage{circuitikz}
  \usepackage{tikz}
\usetikzlibrary{trees}
  
  \usepackage{graphicx}
  %\usetikzlibrary{shapes.arrows,chains}
  \usetikzlibrary{pgfplots.groupplots}
  \usepackage{lscape} % stellenweises Querformat
  \usepackage{verbatim}
  \usepackage{enumitem} 
  \usepackage{dcolumn}%D{,}{,}{Nachkommastellen}


%sonstiges
\usepackage{xspace} %Leerzeichen an Makroenden, falls noetig

%spezielle Symbole
\usepackage{bbm} % fuer mathbbm{1}

%Einstellungen%
\newlength{\longtablewidth}
\setlength{\longtablewidth}{0.675\linewidth}








\usepackage{tikz}
%\usepackage{euler}                                %Nicer numbers






%---- Titel ----%
\title{MIPS 1 in VHDL}

\subtitle{HDL Lab - SS 2015}

\subsubtitle{Bahri Enis Demirtel, Carlos Minamisava Faria,  Lukas Jäger,  Patrick Appenheimer }

%\lowertitleback{\textaccent{}\hfill\today}

\institution{Fachbereich Elektrotechnik und Informationstechnik  \\
    Fachgebiet Integrated Electronic \\Systems Lab}

    
    
\usepgfplotslibrary{polar}
    
    
    % Linienstyle für Diagramme. Variieren sowohl in Farbe als auch Linienstil damit s/w-Druck ohne weiteres möglich
  \pgfplotscreateplotcyclelist{tudLinestyles}{
	{solid,black,thick},
	{densely dashed,tud1b, very thick},
	{densely dotted,tud9b,very thick},
	{loosely dashed,tud3b,very thick},
	{dotted,tud11b,very thick},
	{solid,tud5b,thick},
	{densely dashed,tud7b,very thick},
	{densely dotted,tud4b,very thick},
	{loosely dashed,tud6b,very thick},
	{dotted,tud10b,very thick},
	{solid,tud2b,thick},
}
\pgfplotscreateplotcyclelist{tudLinestylessmooth}{
	{smooth,solid,black,thick},
	{smooth,densely dashed,tud1b, very thick},
	{smooth,densely dotted,tud9b,very thick},
	{smooth,loosely dashed,tud3b,very thick},
	{smooth,dotted,tud11b,very thick},
	{smooth,solid,tud5b,thick},
	{smooth,densely dashed,tud7b,very thick},
	{smooth,densely dotted,tud4b,very thick},
	{smooth,loosely dashed,tud6b,very thick},
	{smooth,dotted,tud10b,very thick},
	{smooth,solid,tud2b,thick},
}

\pgfplotscreateplotcyclelist{tudLinestylessmooth2}{
	{smooth,densely dashed,black,thick},
	{smooth,densely dotted,tud1b, very thick},
	{smooth,densely dotted,tud9b,very thick},
	{smooth,loosely dashed,tud3b,very thick},
	{smooth,dotted,tud11b,very thick},
	{smooth,solid,tud5b,thick},
	{smooth,densely dashed,tud7b,very thick},
	{smooth,densely dotted,tud4b,very thick},
	{smooth,loosely dashed,tud6b,very thick},
	{smooth,dotted,tud10b,very thick},
	{smooth,solid,tud2b,thick},
}


\pgfplotscreateplotcyclelist{tudLinestylesColor}{
	{solid,tud9b,thick},
	{densely dashed,tud3b, very thick},
	{densely dotted,tud1b,very thick},
	{loosely dashed,black,very thick},
	{dotted,tud11b,very thick},
	{solid,tud5b,thick},
	{densely dashed,tud7b,very thick},
	{densely dotted,tud4b,very thick},
	{loosely dashed,tud6b,very thick},
	{dotted,tud10b,very thick},
	{solid,tud2b,thick},
}


%\usepackage[nottoc]{tocbibind}
\usepackage[nottoc,numbib]{tocbibind}
\usepackage{listings}
\usepackage{color}

\definecolor{mygreen}{rgb}{0,0.6,0}
\definecolor{mygray}{rgb}{0.5,0.5,0.5}
\definecolor{mymauve}{rgb}{0.58,0,0.82}

\lstset{ %
  language=VHDL,
  backgroundcolor=\color{white},   % choose the background color; you must add \usepackage{color} or \usepackage{xcolor}
  basicstyle=\tiny,        % the size of the fonts that are used for the code
  breakatwhitespace=false,         % sets if automatic breaks should only happen at whitespace
  breaklines=true,                 % sets automatic line breaking
  captionpos=b,                    % sets the caption-position to bottom
  commentstyle=\color{mygreen},    % comment style
  deletekeywords={...},            % if you want to delete keywords from the given language
  escapeinside={\%*}{*)},          % if you want to add LaTeX within your code
  extendedchars=true,              % lets you use non-ASCII characters; for 8-bits encodings only, does not work with UTF-8
  frame=single,	                   % adds a frame around the code
  keepspaces=true,                 % keeps spaces in text, useful for keeping indentation of code (possibly needs columns=flexible)
  keywordstyle=\color{blue},       % keyword style
  language=Octave,                 % the language of the code
  otherkeywords={*,...},            % if you want to add more keywords to the set
  numbers=left,                    % where to put the line-numbers; possible values are (none, left, right)
  numbersep=5pt,                   % how far the line-numbers are from the code
  numberstyle=\tiny\color{mygray}, % the style that is used for the line-numbers
  rulecolor=\color{black},         % if not set, the frame-color may be changed on line-breaks within not-black text (e.g. comments (green here))
  showspaces=false,                % show spaces everywhere adding particular underscores; it overrides 'showstringspaces'
  showstringspaces=false,          % underline spaces within strings only
  showtabs=false,                  % show tabs within strings adding particular underscores
  stepnumber=2,                    % the step between two line-numbers. If it's 1, each line will be numbered
  stringstyle=\color{mymauve},     % string literal style
  tabsize=2,	                   % sets default tabsize to 2 spaces
  title=\lstname                   % show the filename of files included with \lstinputlisting; also try caption instead of title
}
%---- Dokument ----%
\begin{document}
\pagenumbering{roman}  
\maketitle


%---- Inhaltsverzeichniss ----%  
\tableofcontents
\newpage
\pagenumbering{arabic} 

%---- Bildverzeichnis ----% 
%\listoffigures 
%\addcontentsline{toc}{chapter}{\listfigurename}


%\setcounter{chapter}{-1}

\chapter{Introduction}
In the HDL Lab is a practical exercise of a hardware description language implementation. This semester the task is the implementation of a MIPS I microcontroller in vhdl. 
A requirement to this laboratory is the lecture HDL: Verilog and VHDL by Prof. Dr.-Ing. Klaus Hofmann.\\
MIPS is an acronym for Microprocessor without interlocked pipeline stages.
The MIPS instruction set is a reduced instruction set computer (RISC). There are available references for 32 and 64-bit with many revisions.\\
MIPS was developed in the 80s with the intent to take fully advantage of pipelines. Nowadays this instruction set and structure 
is often used as an hdl first project. Commercially it is used embedded systems such as Windows CE devices, routers, residential gateways and video
consoles such as Nintendo 64, Sony PlayStation, PlayStation 2 and PlayStation Portable. \\

\section{Task}

The objective is to design, implement, synthesize and test a MIPS-I specified processor core on FPGA. 
The hardware description language is VHDL and the target technology is a Virtex5 from Xilinx.
The synthesized microcontroller must be able to run at 50 MHz with a desirable frequency of 200 MHz. 
The microcontroller must use a pipeline of a minimum of 2 and maximum of 6 stages.
The following sub-components are mandatory: ALU, data-path and control-path.\\
A counter test program shall run on the synthesized microcontroller, outputting the counter value
to eight LEDs on the FPGA board.

\chapter{Design}
% input Design sections for no auto new page
This chapter describes the design of a MIPS 1 microcontroller. The microcontroller design is shown \autoref{fig:overview}.

\begin{figure}[h]
\centering{
\resizebox{100mm}{!}{\input{design.pdf_tex}}
\caption{CPU overview}
\label{fig:overview}
}
\end{figure}

The CPU is divided in a control and a datapath blocks. In the datapath there is the pipeline made of five blocks:
\begin{itemize}
 \item IF: Instruction fetch
 \item ID: Instruction decode
 \item Ex: Execution
 \item Me: Memory stage
 \item WB: Writeback
\end{itemize}

The CPU interacts with two external memories and has also a clock and a reset input.
\section{ALU}
This section describes the ALU (arithmetic logic unit). The implementation of the ALU has three parallel data buses consisting of two 32bit input operands and a 32bit result output.
Furthermore, there is an 6bit input for the opcode. The ALU can perform 7 operations: add, sub, and, or, sll, slt and interconnect one of the two inputs to the result output. 
Additionally, there is an zero output flag if the operation results in zero.

\begin{figure}[h!]
  \centering
  \includegraphics[width=0.3\textwidth]{figure/alu.png}
  \caption{ALU}
  \label{fig:ALU}
\end{figure}

	\section{Datapath}
This chapter describes the datapath and its internal elements. The datapath is the component that connects the pipeline components within itself as well as with cpu inputs and outputs and the controlpath.

This MIPS implementation works with a 5 stage pipeline in order to achieve a fast clock. The datapath consists of instruction fetch, instructino decode, execution, memory stage and writeback.
The datapath controls the information flow from one pipeline stage to the next with registers. These writing process occurs on the positive edge of the clock when the pipeline stage input from the controlpath
allows it. That is, the registers forwards information synchronously.
	\section{Controlpath}
The control path is not designed as a finite state machine. Due to its simplicity we chose an approach that is closer to the pipelined structure of the MIPS-Architecture. The control path is built around a 32-bit wide 4-deep shift register. When the memory returns the instruction to the instruction-decode-stage of the datapath, it is also fed into the shift register and propagates in the following clock cycles. To the stages of this shift register we attatched a decoder that produces control signals for one stage of our datapath. Each stage of the shift register matches exactly one stage of the datapath, except for the first controller stage which handles instruction fetch and instruction decode. The second stage returns the control signals for the execution stage, the third stage is mapped to the memory stage and the fourth stage controls the datapath's writeback stage. In case of any stalls the propagation of the instructions through the shift register is halted.
\begin{figure}[h!]
  \centering
  \includegraphics[width=1.0\textwidth]{figure/control.png}
  \caption{Control path}
  \label{fig:control}
\end{figure}
\subsection{Decoder for the instruction-fetch- and instruction-decode-stage}
The decoder attatched to the first stage of the shift register distinguishes between the opcodes of the given instruction to determine the control signals. 
If the opcode is 000000, an R-type-instruction is assumed and the instruction-decode-stage's destination register control signal is set to 0 to use the R-type destination register. 
The shift multiplexor is set to 0 as well to use the R-type-instructions shift field. For the settings of the pc's multiplexor in the instruction-fetch-stage, 
the other bits of the instruction are evaluated. If they make the instruction a Jump-Register-instruction, the signal is set to 1 so that the new program counter calculated by 
the branch logic is used instead of the previous value incremented by 4. In every other case, this multiplexor's control signal is set to zero because all other R-type-instructions 
do not modify the program counter.

All non-R-type-instructions are treated as I-type-instructions by the decoder. This means, that the control signal for the shift-multiplexor is set to 0 
for all instructions except the LUI-Instruction, which needs a shift of 16. Therefore, the signal is set to 1. A more complex decision has to be made for the program counter multiplexor. 
All instructions that influence the program's control flow (Branch- and Jump-Instructions) produce an output of 1, all other instructions return 0 and the program counter is incremented by 4. 
The multiplexor for the destination register is always set to 2 to use the I-type-instruction destination field. This is again ignored, when a J-type-instruction reaches the further stages, so no damage is done.
\subsection{Decoder for the execution-stage}
The decoder for the execution stage has to set the signals that select the operands of the operation to be executed and set the kind of operation the ALU has to perform. Since the instruction's opcode is in
hardly any way connected to the executed operation, the decoder logic is a little more difficult. It groups all operations of the example code that perform an addition of a register's value to the immediate 
value of the instruction (ADDIU, LW, SW, SB, LBU) and sets the control signals to 2 for the ALU's a-operand, 1 to forward the immediate field to the ALU and the ALU's operation code itself is set to 20
for addition. 

For the LUI-instruction, the ALU's a-operand-multiplexor is set to 0 for a shift of 16, the b-operand-multiplexor is set to 1 for the immediate-field and the ALU itself is perfoming the shift
operation. All other immediate-operations get the a-operand-multiplexor set to 2, b set to 1 and the ALU operation code set according to the instruction. The only supported R-type operation, SLT gets a set to 2,
b set to 0 and the ALU set for set-less-than-operations. All other R-Type instructions are treated like NOOP-instructions: A set to 2, b set to 0 and the ALU-operation set to a left-shift. 
Although the datapath would be able to support more operations, the datapath is limited to the described instructions and has to be expanded for a full MIPS instruction set.
\subsection{Decoder for the memory stage}
The memory stage's decoder is more simple than the execution stage's decoder. It just has to evaluate whether the instruction is a store-instruction,
a load-instruction or any other instruction. In case of a load instruction the multiplexor signal for the memory access has to be set to 1 to let the result of the memory access get to the writeback stage. 
In all other cases, this signal is set to 0 to just forward the signal coming from the execution stage. The read or write masks are set according to the instruction stored in the shift register stage.
A SW-instruction for example produces an output of F for the write mask and an LBU-instruction returns a read mask of 1. All other instructions have read and write masks of 0.
\subsection{Decoder for the writeback stage}
The decoder for the writeback stage is the simplest of the whole controller because it controls just one signal that enables the register bank. It distinguishes between the commands that write back to the register bank bank (currently supported: LUI, ADDIU, LW, LBU, SLTI, SLT, ANDI, ORI) and sets the \em enable\_ regs\em -signal to for them, and the other instructions, where the register bank is disabled by setting the signal to zero. 
\chapter{Evaluation}
The cpu evaluation is done with Modelsim from MentorGraphics. The simulations provide a timed analysis of the code. It provides information
about the timing relations and allows for bug identification still in a simulation environment, without the need of a complete synthesis
and programming of the FPGA. This is a powerful tool to speed up the development process evaluating the design in an early stage.

Individual testbenches test each separate cpu components on all hierarchical levels up to the complete cpu.
All component's simulation passed, including the complete cpu with a simulated perfect memory. Furthermore the
simulation with a simulated real memory passed. The tests prove the complete implementation up to real conflict cases of instruction and data access stalls.

For the implementation on the FPGA a hdllab code was prepared with the cpu, memory, UART and pll components as well as LEDs, clock and reset interface
with the FPGA already integrated. The cpu passed this simulation with the counter program, outputting the counter value to the LEDs output. 

The evaluation phase was successful, proving exhaustively the correct behaviour of each component separately and as a piece of the whole cpu. The functional
test of a counter program serve also as a prove of concept. 
\chapter{Synthesis}
The design was synthesized using the Xilinx ISE suite and the \em hdllab\em \ configuration. The target device for synthesis is a Xilinx Virtex-5 FPGA. Since MIPS is a rather compact architecture, the target FPGA provides a lot of programmable resources and a clock frequency of 50 MHz is a mandatory requirement, we opted for the design strategy that promised the highest maximum clock frequency and did not care much about the used space. However a comparison to the results of the balanced synthesis strategy showed no significant differences.\\
The synthesis result can be operated with a clock frequency up to 73.27 MHz and thus with a minimal clock period of 13.65 ns, so the frequency requirements to the project can be fulfilled. The maximum delay is caused by the instruction decode stage, which is the most complex part of the processor and it is not surprising that it is limiting the maximum frequency of the whole design. The minimum input arrival time of the design is 9.82 ns, the maximum output time required after clock cycle is 3.27 ns and the maximum combinatorial path delay is 1.15 ns.\\
Although the resource usage was not an optimization goal, the synthesis result is rather compact. It uses 3579 slice registers and 4090 slice look-up-tables, which is 7\% of the FPGA's total available slice registers and 9\% of the look-up-tables. Combined, 7281 logic slices are used by the design. 388 of these slices are fully used, 3702 of them contain an unused flip-flop and 3191 contain an unused look-up-table. This means that 95\% of the used slices are not fully utilized. An area-optimizing synthesis strategy could improve this rather bad utilization percentage values but the design still takes only 16 \% of the available logic slices on the FPGA so there is no urgent reason to shrink the size of the result at the cost of making it slower. In fact, our design could surely be used on a more low-end FPGA at about the same speed. On the other hand the free space on the Virtex-5 FPGA could easily be used to build a MIPS-based System-on-chip on top of our project. Of course for full MIPS compatibility the control path and the ALU have to be expanded, but there is plenty of space for it.\\
13 IO ports of the FPGA are used. That makes eight for the LEDs, two for UART, two more for the clock and one for the reset. Of the available 640 IOBs, the 13 used ones make only 2\%.\\
The synthesis result was tested with the original counter assembler code, waiting 16 clock cycles before the memory cell is incremented by one. The UART-interface was compiled with the frequency parameter configured for 50 MHz to meet the actual board's clock frequency and the \em hdllab\em -project uses the given memory for FPGAs, not the memory used for simulation. The LEDs flashed as they were supposed to be but 16 clock cycles are to fast for a human eye to detect, so we used an attached oscilloscope to test for the correct blinking sequence. The counter version that should make the LEDs blink perceivably for the human eye however did not work. We suppose that this modified counter contains MIPS-instructions that are somehow not supported by our implementation. Although we searched for new instructions and added code to the controller (the datapath should be able to process any MIPS-instruction and needs no further modification) to process them, the example did not make the LEDs blink. But since the requirement to run the original counter was fulfilled and the time was pressing, no further efforts were made to make this functionality possible.
\chapter{Conclusion}
In this laboratory a 32-bit MIPS I CPU with restricted instruction set is successfully designed in vhdl, implemented, synthesized and test a MIPS-I specified processor core on FPGA. 
The required components of ALU, data-path and control-path are present.
The implementation on the FPGA passed a functional test with the counter sample program, running at the desired speed of 50 MHz and blinking the LEDs with the 8-bit counter value.

The CPU design comprises a control-path and a data-path with five pipeline stages.  Each component was designed and simulated separately and
together up to as a complete CPU. The simulations in Modelsim included test cases for a perfect memory and a real one with instruction and data stalls.
The simulation of a functional test with the program counter passed outputting the counter value as an 8-bit LED array.

The synthesis is done with Xilinx ISE for the FPGA Virtex5. With configurations for the fastest clock, the synthesis reports a maximum running
frequency of over 70\,MHz. The clock configuration for the on board test is set at 50 MHz. The synthesized code passes the functional test with
the counter program on the Virtex5. This counter uses the planned instruction set.

This work shows an implementation of a restricted MIPS instruction set. The expansion of this instructions is expected to lead to a full MIPS 32-bit compliant microcontroller.
	%************* Verzeichnisse ************
	\clearpage
\bibliographystyle{unsrt}
\bibliography{myref}
	  %\bibliography{myref} %.bib-Datei, falls biblatex benutzt wird
	%\printbibliography %Literatur
	%\listoffigures %Abbildungen
	%****************************************
\clearpage
	%****************************************
	%Anhang
	%****************************************	
	\appendix
	%\addcontentsline{toc}{chapter}{Appendix}
	%\include{code}
\end{document}
