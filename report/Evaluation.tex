\chapter{Evaluation}
The CPU evaluation is done with Modelsim from MentorGraphics. The simulations provide a timed analysis of the code. It provides information
about the timing relations and allows for bug identification still in a simulation environment, without the need of a complete synthesis
and programming of the FPGA. This is a powerful tool to speed up the development process evaluating the design in an early stage.

Individual test-benches test each separate CPU components on all hierarchical levels up to the complete CPU.
All component's simulation passed, including the complete CPU with a simulated perfect memory. Furthermore the
simulation with a simulated real memory passed. The tests prove the complete implementation up to real conflict cases of instruction and data access stalls.

For the implementation on the FPGA a hdllab code was prepared with the CPU, memory, UART and pll components as well as LEDs, clock and reset interface
with the FPGA already integrated, as shown in \autoref{fig:hdllab}. The CPU passed this simulation with the counter program, outputting the counter value to the LEDs output. 

\begin{figure}[h!]
\begin{center}
 \tikzstyle{every node}=[draw=black,thick,anchor=west]
\tikzstyle{selected}=[draw=red,fill=red!30]
\tikzstyle{optional}=[dashed,fill=gray!50]
\begin{tikzpicture}[ grow via three points={
        one child at (0.8,-0.7) and two children at (0.8,-0.7) and (0.8,-1.4)
    },
    edge from parent path={
        ($(\tikzparentnode\tikzparentanchor)+(.4cm,0pt)$) |- (\tikzchildnode\tikzchildanchor)
    },
    growth parent anchor=west,
    parent anchor=south west,]
  \node {hdllab}
  child { node {cpu}	
    child { node {controlpath}}		
    child { node {datapath}
      child { node {Instruction Fetch}}
      child { node {Instruction Decode}}      
      child { node  {Execution}
	%child { node {ALU}}
	}
      child { node {Memory Stage}}
      child { node {Writeback}}
    }    
    }
    child [missing] {}				
    child [missing] {}				
    child [missing] {}
    child [missing] {}				
    child [missing] {}				
    child [missing] {}			
    child [missing] {}
  child { node {pll}}	
  child { node {memory}}	
  child { node {uart}}	
  ;
\end{tikzpicture}
\caption{CPU component structure}
\label{fig:hdllab}
\end{center}
\end{figure}
The evaluation phase was successful, proving exhaustively the correct behavior of each component separately and as a piece of the whole CPU. The functional
test of a counter program serve also as a prove of concept. 