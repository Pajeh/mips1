\chapter{Evaluation}
The cpu evaluation is done with Modelsim from MentorGraphics. The simulations provide a timed analysis of the code. It provides information
about the timing relations and allows for bug identification still in a simulation environment, without the need of a complete synthesis
and programming of the FPGA. This is a powerful tool to speed up the development process evaluating the design in an early stage.

Individual testbenches test each separate cpu components on all hierarchical levels up to the complete cpu.
All component's simulation passed, including the complete cpu with a simulated perfect memory. Furthermore the
simulation with a simulated real memory passed. The tests prove the complete implementation up to real conflict cases of instruction and data access stalls.

For the implementation on the FPGA a hdllab code was prepared with the cpu, memory, UART and pll components as well as LEDs, clock and reset interface
with the FPGA already integrated. The cpu passed this simulation with the counter program, outputting the counter value to the LEDs output. 

The evaluation phase was successful, proving exhaustively the correct behaviour of each component separately and as a piece of the whole cpu. The functional
test of a counter program serve also as a prove of concept. 